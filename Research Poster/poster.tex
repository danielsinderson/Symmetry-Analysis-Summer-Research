\documentclass[25pt, a0paper, landscape]{tikzposter} % See Section 3
\title{Comparing Symmetries of Nonlinear PDEs and their Linearizations} \institute{Southern Oregon University} % See Section 4.1
\author{William Helman \& Daniel Sinderson} \titlegraphic{Logo}
\usetheme{Basic} % See Section 5
\begin{document}
	\maketitle % See Section 4.1
	\block{Introduction and Background}{
	\textbf{OBJECTIVE:} Our research objective for this project was to calculate the symmetry groups for the Reduced Gibbons-Tsarev equation and the Born-Infeld equation, and then determine if they were the same as the symmetry groups for their linearizations.
	\\\\ \textbf{THE EQUATIONS:} Both the Born-Infeld and Reduced Gibbons-Tsarev equations are nonlinear, second-order partial differential equations in two independent variables. The Born-Infeld equation originally popped up in the context of nonlinear electrodynamics and is still found today in string theory in the description of the action of open strings. The Reduced-Gibbons Tsarev equation arose in the context of parametrizations of the Benney moment equations with finitely many (two) dependent variables. It finds use in the study of dispersionless systems.
	\\\\ \begin{center}\Large{Born-Infeld: $(u_y^2 - 1)u_{xx} - 2u_xu_yu_{xy}+(u_x^2 + 1)u_{yy}=0$ \hspace{5in} Reduced Gibbons-Tsarev: $u_{xx}-u_yu_{xy}+u_xu_{yy}=0$
	\vspace{1in} \\Linearized Born-Infeld: $u_{xy}+2(\frac{xu_x-yu_y}{x^2-y^2})=0$} \hspace{5in} Linearized Reduced Gibbons-Tsarev: $u_{xy}+\frac{u_x-u_y}{x-y}=0$ \end{center}
	} % See Section 4.2
	\begin{columns} % See Section 4.4
		\column{0.25} % See Section 4.4
		\block{Methods}{
			In order to accomplish our objective we made use of both mathematical and computational tools. 
			\\\\ \textbf{Mathematical Tools}\\ On the mathematical side, the first step in the process is to construct Lie's Invariance Condition for the equation.
			\\\\ \textbf{Computational Tools}\\ On the computational side, we used the computer algebra system Maple. This allowed us to do the heavy number-crunching and symbol manipulation required by the mathematical tools we used in minutes rather than days.
			\\\\ \textbf{Approach}
			\begin{enumerate}
				\item Define the total derivative operator in Maple.
			\end{enumerate}				
		}
		\column{0.5}
		\block{Conclusion}{Conclusion}
		\column{0.25}
		\block{Results}{Results}
	\end{columns}
\end{document}