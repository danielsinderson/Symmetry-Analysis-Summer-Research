\documentclass{beamer}

\usepackage{amsmath}

%\usetheme{}

\mode<presentation>

%Information to be included in the title page:
\title{Comparing the Classical and Nonclassical Symmetries of Nonlinear Partial Differential Equations}
\author{William Helman and Daniel Sinderson}
\institute{Southern Oregon University}
\date{2023}

\begin{document}

\frame{\titlepage}



\begin{frame}
    \frametitle{Introduction}
    Our research objective for this project was to calculate the classical and nonclassical symmetry groups for the reduced Gibbons-Tsarev equation and the Born-Infeld equation and compare them.
\end{frame}



\begin{frame}
    \frametitle{What is a symmetry?}
    \begin{definition}<1->
        A symmetry is a transformation that leaves an object invariant.
    \end{definition}
    \vspace*{0.5in}
    \begin{definition}<2->
        A symmetry is a change that doesn't change anything.
    \end{definition}
\end{frame}


\begin{frame}
    \frametitle{What is a symetry?}
    Let's see this in action using the simple linear equation $x-y=0$.\\
    \begin{center}
        \includegraphics[width=9cm]{y=x.png}
    \end{center}
\end{frame}

\begin{frame}
    \frametitle{What is a symmetry?}
    \begin{example}[A Non-Example]
        \begin{itemize}
            \item For our first transformation, let's define new variables\\ $\bar{x}=x+1$ and $\bar{y}=y$.
            \item Now we rewrite our equation using these new variables. \begin{equation*}
                \begin{aligned}
                    \bar{x}-\bar{y} &= 0 & \text{by definition} \\
                    x+1-y &= 0 & \text{by substitution} \\
                    y &= x+1 & \text{by rewriting in slope-intercept form}
                \end{aligned}
            \end{equation*}
            \item This is transformation is not a symmetry: \\ $x-y+1\ne x-y$
        \end{itemize}        
    \end{example}
\end{frame}


\begin{frame}
    \frametitle{What is a symmetry?}
    \framesubtitle{A Transformation that is a Symmetry}
    \begin{example}[2]<1->
        \begin{itemize}
            \item Let's define some new variables again\\ $\bar{x}=x+1$ and $\bar{y}=y+1$.
            \item Now we rewrite our equation using these new variables. \begin{equation*}
                \begin{aligned}
                    \bar{x}-\bar{y} &= 0 & \text{by definition} \\
                    (x+1)-(y+1) &= 0 & \text{by substitution} \\
                    (x-y)+(1-1) &= 0 & \text{by algebra} \\
                    x-y &= 0 & \text{by algebra} \\
                    y &= x & \text{by rewriting in slope-intercept form}
                \end{aligned}
            \end{equation*}
            \item This transformation is a symmetry: \\ $x-y = x-y$
        \end{itemize}        
    \end{example}
\end{frame}


\begin{frame}
    \frametitle{What is a symetry?}
    The graphs of our three equations.\\
    \begin{center}
        \includegraphics[width=9cm]{y=x+1.png}
    \end{center}
\end{frame}



\begin{frame}
    \frametitle{What is a symmetry?}
    \begin{Large}
        Who cares?
    \end{Large}
    \vspace*{0.25in}
    \begin{itemize}
        \item Symmetries help us understand and solve equations that we wouldn't normally be able to.\pause
        \item Symmetries encode physically meaningful aspects of equations, like conservation laws in physics.\pause
        \item They're cool.
    \end{itemize}
\end{frame}



\begin{frame}
    \frametitle{What is a Differential Equation?}

\end{frame}



\begin{frame}
    \frametitle{The History of the Born-Infeld and the reduced Gibbons-Tsarev Equations}

\end{frame}



\begin{frame}
    \frametitle{The Classical Symmetries of the Born-Infeld and the reduced Gibbons-Tsarev Equations}

\end{frame}



\begin{frame}
    \frametitle{The Nonclassical Symmetries of the Born-Infeld and the reduced Gibbons-Tsarev Equations}

\end{frame}



\begin{frame}
    \frametitle{Future Work: Does Integrability Imply Equivalence of Classical and Nonclassical Symmetries?}

\end{frame}



\begin{frame}
    \frametitle{Future Work: Does Equivalence of Classical and Nonclassical Symmetries Imply Integrability?}

\end{frame}



\end{document}